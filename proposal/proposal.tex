% Template: https://www.usenix.org/conferences/author-resources/paper-templates
\documentclass[letterpaper,twocolumn,10pt]{article}
\usepackage{usenix2019_v3}

% to be able to draw some self-contained figs
\usepackage{tikz}
\usepackage{amsmath}

% inlined bib file
\usepackage{filecontents}

%-------------------------------------------------------------------------------
\begin{document}
%-------------------------------------------------------------------------------

%don't want date printed
\date{}

% make title bold and 14 pt font (Latex default is non-bold, 16 pt)
\title{\Large \bf Usage of Authoritarian Language in Reddit Moderators}

\author{ % alphabetical
{\rm David Li}\\
dl1@terpmail.umd.edu
\and
{\rm Kyle Lin}\\
klin1215@terpmail.umd.edu
\and
{\rm Leo Wang}\\
leowang@terpmail.umd.edu
}

\maketitle

%-------------------------------------------------------------------------------
\section{Motivation}
%-------------------------------------------------------------------------------

What is it you are trying to solve?
Why is it important?
What is the question that this is trying to solve?

%-------------------------------------------------------------------------------
\section{Prior Work}
%-------------------------------------------------------------------------------

Nothing in depth at this time (that comes with the Intermediate report writeup), but at least a perusal of what you think are the most relevant papers.

%-------------------------------------------------------------------------------
\section{Methodology}
%-------------------------------------------------------------------------------

Low-level details aren't strictly needed at this time, but what broadly do you intend to do? Create a new tool? Collect a new dataset? Analyze and existing dataset?

%-------------------------------------------------------------------------------
\section{Evaluation}
%-------------------------------------------------------------------------------

What are your metrics for success? Performance? Privacy? Predictive ability of a new model?
Will you be comparing to any other prior approaches?

%%%%%%%%%%%%%%%%%%%%%%%%%%%%%%%%%%%%%%%%%%%%%%%%%%%%%%%%%%%%%%%%%%%%%%%%%%%%%%%%
\end{document}
%%%%%%%%%%%%%%%%%%%%%%%%%%%%%%%%%%%%%%%%%%%%%%%%%%%%%%%%%%%%%%%%%%%%%%%%%%%%%%%%
